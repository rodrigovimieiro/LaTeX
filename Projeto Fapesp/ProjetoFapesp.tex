%% Projeto Fapesp
%% 
%% Nome: Rodrigo de Barros Vimieiro
%% E-mail: rodrigo.vimieiro@gmail.com
%%
%% Universidade de São Paulo - São Carlos
%% Laboratório de Visão Computacional - LAVI
%%
%% Data: 06/11/2017


% ------------------------------------------------------------------------
% ------------------------------------------------------------------------
% eesc: Modelo de Trabalho Acadêmico (tese de doutorado, dissertação de
% mestrado e trabalhos monográficos em geral) em conformidade com 
% ABNT NBR 14724:2011. Esta classe estende as funcionalidades da classe
% abnTeX2 elaborada de forma a adequar os parâmetros exigidos pelas 
% normas USP e do departamento de elétrica da Escola de Engenharia 
% de São Carlos - USP.
% ------------------------------------------------------------------------
% ------------------------------------------------------------------------

% ------------------------------------------------------------------------
% Opções:
% 	tesedr:     Formata documento para tese de doutorado
%	qualidr:    Formata documento para qualificação de doutorado
% 	dissertmst: Formata documento para dissertação de mestrado
% 	qualimst:   Formata documento para qualificação de mestrado
% ------------------------------------------------------------------------
\documentclass[tesedr]{eesc}

% ---
% PACOTES
% ---

% ---
% Pacotes fundamentais 
% ---
\usepackage{cmap}				% Mapear caracteres especiais no PDF
\usepackage{lmodern}			% Usa a fonte Latin Modern			
\usepackage{makeidx}           	% Cria o indice
\usepackage{hyperref}  			% Controla a formação do índice
\usepackage{lastpage}			% Usado pela Ficha catalográfica
\usepackage{indentfirst}		% Indenta o primeiro parágrafo de cada seção.
\usepackage{nomencl} 			% Lista de simbolos
\usepackage{graphicx}			% Inclusão de gráficos
% ---

% ---
% Pacotes adicionais, usados apenas no âmbito do Modelo eesc
% ---
\usepackage[printonlyused]{acronym}
\usepackage[table]{xcolor}
% ---


% ---
% Informações de dados para CAPA e FOLHA DE ROSTO
% ---
%
% Título:
%	1. Título em português
%	2. Título em inglês
\titulo{Título em português}{}
%
% Autor:
%	1. Nome completo do autor
%	2. Formato de nome para bibliografia
\autor{Nome completo do autor}{Sobrenome, Nome}
%
% Cidade
\local{São Carlos}
% Ano de defesa
\data{2018}
% Área de concentração da pesquisa
\areaconcentracao{Processamento de Sinais e Instrumentação}
% Nome do orientador
\orientador{Nome completo do Orientador}
% Nome do coorientador
%\coorientador{}
% ---

% ---
% compila o indice
% ---
\makeindex
% ---

% ---
% Compila a lista de abreviaturas e siglas
% ---
\makenomenclature
% ---

% ---
% Inserir ficha catalográfica
%
% Caso o comando \inserirfichacatalografica seja definido, a ficha catalográfica
% será inserida atrás da folha de rosto. Caso contrário a página será deixada em
% branco.
%
% CUIDADO: Esta opção deve ser preenchida antes do comando \maketitle
% ---
%\inserirfichacatalografica{fichaCatalografica.pdf}
% ---

% ---
% Inserir folha de aprovação
%
% Caso o comando \inserirfolhaaprovacao seja definido, a a folha de aprovação
% será inserida. Além disso, conforme Resolução CoPGr 5890, as informações 
% de rodapé são inseridas apropriadamente na folha de rosto.
%
% CUIDADO: Esta opção deve ser preenchida antes do comando \maketitle
% ---
%\inserirfolhaaprovacao{folhaAprovacao.pdf}
% ---

% ----
% Início do documento
% ----

\begin{document}

% ----------------------------------------------------------
% ELEMENTOS PRÉ-TEXTUAIS
% ----------------------------------------------------------
\pretextual

% ---
% Insere Capa, Folha de rosto, Ficha catalográfica (se inserida)
% e folha de aprovação (se inserida).
% ---
\maketitle

%%%%%%%%%%%%%%%%%%%%%%%%%%%%%%%%%%%%%%%%%%%%%%%%%%%%%%%%%%%%%%%%%%%%%%%%%%%%%%%%%%%%%%%%%%%%%%%%%%%%%%%%%%%%%%%%%%%%%%%%%%%%%%													   Preâmbulo															%
%%%%%%%%%%%%%%%%%%%%%%%%%%%%%%%%%%%%%%%%%%%%%%%%%%%%%%%%%%%%%%%%%%%%%%%%%%%%%%%%%%%%%%%%%%%%%%%%%%%%%%%%%%%%%%%%%%%%%%%%%%%%%
\begin{preambulo}
	
O projeto de pesquisa\footnote{\url{http://www.fapesp.br/253}} deve ser apresentado de maneira clara e resumida, ocupando no máximo 20 páginas digitadas em espaço duplo. Para propostas encaminhadas através do Sistema de Apoio a Gestão (SAGe), deve anexar documento tipo DOC ou PDF de até 5Mb.

\noindent Deve compreender:


\begin{enumerate}

\item Resumo (máximo 20 linhas);

\item Introdução e justificativa, com síntese da bibliografia fundamental;

\item Objetivos;

\item Plano de trabalho e cronograma de sua execução;

\item Material e métodos;

Forma de análise dos resultados.
\end{enumerate}

A FAPESP espera que o candidato participe intensamente na redação do projeto. A responsabilidade pelo projeto é do orientador, mas o candidato deve estar preparado para discuti-lo e analisar os resultados. No caso de bolsas de IC, quando houver projetos de IC associados a um mesmo projeto de pesquisa do orientador, é imprescindível que para cada candidato a IC haja uma parte bem definida designada, mesmo que faça referência a outros projetos de IC em solicitação ou em andamento. 

\end{preambulo}


%%%%%%%%%%%%%%%%%%%%%%%%%%%%%%%%%%%%%%%%%%%%%%%%%%%%%%%%%%%%%%%%%%%%%%%%%%%%%%%%%%%%%%%%%%%%%%%%%%%%%%%%%%%%%%%%%%%%%%%%%%%%%%												  RESUMO e ABSTRACT															%
%%%%%%%%%%%%%%%%%%%%%%%%%%%%%%%%%%%%%%%%%%%%%%%%%%%%%%%%%%%%%%%%%%%%%%%%%%%%%%%%%%%%%%%%%%%%%%%%%%%%%%%%%%%%%%%%%%%%%%%%%%%%%
% Resumo em português
\begin{resumo}{palavra-chave1, palavra-chave2, palavra-chave3}
	
Resumo (máximo 20 linhas);

\end{resumo}

% Resumo em inglês
%\begin{abstract}{palavra-chave2. palavra-chave2. palavra-chave2}
%This is the english abstract.
%	
%\end{abstract}


%%%%%%%%%%%%%%%%%%%%%%%%%%%%%%%%%%%%%%%%%%%%%%%%%%%%%%%%%%%%%%%%%%%%%%%%%%%%%%%%%%%%%%%%%%%%%%%%%%%%%%%%%%%%%%%%%%%%%%%%%%%%%%												  	DISSERTAÇÃO																%
%%%%%%%%%%%%%%%%%%%%%%%%%%%%%%%%%%%%%%%%%%%%%%%%%%%%%%%%%%%%%%%%%%%%%%%%%%%%%%%%%%%%%%%%%%%%%%%%%%%%%%%%%%%%%%%%%%%%%%%%%%%%%


% ---
% inserir lista de ilustrações
% ---
%\listailustracoes
% ---

% ---
% inserir lista de tabelas
% ---
%\listatabelas
% ---

% ---
% inserir lista de abreviaturas e siglas
% ---
\listasiglas{abrev/Abreviaturas}
% ---

% ---
% inserir o sumario
% ---
\sumario
% ---

% ----------------------------------------------------------
% ELEMENTOS TEXTUAIS
% ----------------------------------------------------------
\mainmatter

% ----------------------------------------------------------
% Introdução
% ----------------------------------------------------------

\chapter[Introdução]{Introdução}\label{Introdução}

Introdução e justificativa, com síntese da bibliografia fundamental;







  

  

\chapter[Objetivo]{Objetivo}\label{Objetivo}

Objetivos



  

  

\chapter[Plano de Trabalho \& Cronograma]{Plano de Trabalho \& Cronograma}\label{PlanoTrabalhoCronograma}

Plano de trabalho e cronograma de sua execução;



  

  

\chapter[Materiais \& Métodos]{Materiais \& Métodos}\label{MateriaisMétodos}

Materiais e métodos;


  

  

\chapter[Forma de análise dos resultados]{Forma de análise dos resultados}\label{FormaAnaliseResultados}

Forma de análise dos resultados.





  

  



% ---
% Finaliza a parte no bookmark do PDF, para que se inicie o bookmark na raiz
% ---
\bookmarksetup{startatroot}% 
% ---

% ---
% Conclusão
% ---

%\chapter*[Conclusão]{Conclusão}
%\addcontentsline{toc}{chapter}{Conclusão}

%\lipsum[31-33]

% ----------------------------------------------------------
% ELEMENTOS PÓS-TEXTUAIS
% ----------------------------------------------------------
\postextual

% ----------------------------------------------------------
% Referências bibliográficas
% ----------------------------------------------------------
\bibliography{bib/referencias}


% ----------------------------------------------------------
% Apêndices
% ----------------------------------------------------------
% ---
% Inicia os apêndices
% ---
%\begin{apendicesenv}
%% Imprime uma página indicando o início dos apêndices
%\partapendices
% ----------------------------------------------------------
% Incluir Apêndice
% ----------------------------------------------------------

%\include{cap/Apen}
%
%\end{apendicesenv}
% ---

% ----------------------------------------------------------
% Anexos
% ----------------------------------------------------------
% ---
% Inicia os anexos
% ---
%\begin{anexosenv}
% Imprime uma página indicando o início dos anexos
%\partanexos
% ----------------------------------------------------------
% Incluir Anexo
% ----------------------------------------------------------


%\end{anexosenv}


\end{document}


